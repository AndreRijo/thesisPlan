%!TEX root = ../template.tex
%%%%%%%%%%%%%%%%%%%%%%%%%%%%%%%%%%%%%%%%%%%%%%%%%%%%%%%%%%%%%%%%%%%
%% chapter1.tex
%% NOVA thesis document file
%%
%% Chapter with introduction
%%%%%%%%%%%%%%%%%%%%%%%%%%%%%%%%%%%%%%%%%%%%%%%%%%%%%%%%%%%%%%%%%%%

\typeout{NT FILE chapter1.tex}%

\chapter{Introduction}
\label{cha:introduction}

``The Introduction should include a short summary of the major issues be- hind the proposed research, and provide the context of those issues within a broader background (not only in academic, but also in industrial and social terms, if possible). It may already provide a glimpse of the problems and challenges you intend to solve and the kind of results you expect to obtain, in general terms. After reading the Introduction, anyone should get an intuitive grasp of what the research issues are and why they are relevant.''

- Motivate the research topic

- What are the challenges/issues we are trying to solve? Provide a broad context (academic, industrial, social)

- Can provide a glimpse of problems and challenges that are intended to be solved; also possible expected results, in general terms

- The idea is that after reading the Introduction, anyone should know what are the research issues/challenges and why they are relevant
