%!TEX root = ../template.tex
%%%%%%%%%%%%%%%%%%%%%%%%%%%%%%%%%%%%%%%%%%%%%%%%%%%%%%%%%%%%%%%%%%%%
%% chapter4.tex
%% NOVA thesis document file
%%
%% Chapter with lots of dummy text
%%%%%%%%%%%%%%%%%%%%%%%%%%%%%%%%%%%%%%%%%%%%%%%%%%%%%%%%%%%%%%%%%%%%

\typeout{NT FILE chapter4.tex}%

\chapter{Work Plan}
\label{cha:work_plan}

%Grant started on 1st december 2019, 42 months.
%So originally ends at 1st june 2023. Add another 6 months. 1st december 2023
%So another 2-3 months of covid... 1st march 2024?

\todo{Do I need to mention here the challenges/difficulties to address? I feel like that was already mentioned in the Research Statement.}

In this section we describe how we plan to make usage of the remaining time until the end of the PhD.
Table \ref{table:work_plan} shows the distribution of time for each task left.
We have described the challenges of each task in Chapter \ref{cha:research_statement}.

\todo{I will make some scheme similar to the one done for the grant when this is more well defined. The current table is just temporary.}
\todo{Should I in the plan mention about our submission to VLDB? Like have that be part of the table/figure?}
\todo{A concern of mine is that adding ``consistency levels'' may imply changes to dynamic partitioning.}

%\begin{center}
%	\begin{tabular}{c | c}
%		\hline
%		Time frame & Activity \\
%		\hline
%		June 2022 to September 2022 & garbage collection for version management \\
%		September 2022 to November 2022 & 
%		Implementation of alternative version management solution \\
%		September 2022 to January 2023 & Evaluation of garbage collection and version management solutions \\
%		November 2022 to May 2023 & Dynamic partitioning \\
%		March 2023 to June 2023 & Evaluation of PotionDB with dynamic partitioning \\
%		February 2023 to November 2023 & Consistency levels \\
%		July 2023 to December 2023 & Evaluation of consistency levels + final PotionDB evaluation \\
%		September 2023 to March 2024 & Writing of PhD thesis and final publications \\
%		\hline
%	\end{tabular}
%	\label{table:work_plan}
%\end{center}

\setlength{\tabcolsep}{18pt}
\renewcommand{\arraystretch}{1.2}
\begin{table}[h!]
	\centering
	{\rowcolors{2}{gray!10!white!90}{white!100}
	\begin{tabular} {ll}%{|p{4cm}|p{9cm}|}
		\toprule
		\textbf{Time Frame} & \textbf{Activities} \\
		\midrule
		July 2025 -- August 2025 & Possible revision of VLDB article \\
		July 2025 -- September 2025 & Invariants + PotionDB SQL \\
		September 2025 -- January 2026 & Dynamic Partitioning \\
		November 2025 -- January 2026 & Consistency Levels \\
		February 2026 -- March 2026 & GC + Alternative Version Management \\
		March 2026 -- April 2026 & Ph.D. Proposal Preparation \\
		April 2026 -- September 2026 & Ph.D. Dissertation \\
		\bottomrule
	\end{tabular}}
	\label{table:work_plan}
	\caption{Tentative time line of activities}
	\vspace*{-0.8em}
\end{table}

%TODO: Re-read and compact (all below).
We submitted a paper to VLDB with our core contributions done so far, which is currently under review.
Given that VLDB is a A* conference, a lot of work has been done in the context of evaluating PotionDB, namely fully implementing TPC-H's benchmark \cite{tpch}, flexibility on client/server settings and functionalities, etc.
We have also implemented micro benchmarking and YCSB's benchmark \cite{ycsb}.
As such, the next evaluating steps will be faster, as the evaluation foundation is already there.
We run our experiments in the Grid'5000 testbed \cite{Grid5000}, enabling us to use multiple machines for distributed experiments.
We simulate latency from geo-distribution with tc \cite{tc}, based on real measurements \cite{AWSLatency}.
%A lot of work has already been done in the context of evaluating PotionDB, namely by implementing TPC-H's benchmark and other test clients.
%We believe this will make the next evaluating steps quicker and smoother, as the testing setup can be leveraged on for the next steps.
%Our evaluation will be done in the Grid'5000 testbed, which allows to make usage of multiple machines to run instances of PotionDB on.
%Latency from geo-distribution can be simulated by imposing latency on the links, which was already done in our experiments.
%
%\todo{Do I need to make any special reference to Grid 5000?}

There is considerable overlap between Dynamic Partitioning and Consistency Levels.
Initially we will focus solely on Dynamic Partitioning with the current consistency options offered by PotionDB.
We foresee that introducing extra consistency levels may have implications on the dynamic partitioning, specially strong consistency, hence the overlap.

%There is some considerable overlap between dynamic partitioning and consistency levels.
%This happens as supporting different consistency models and/or invariants may have impacts on how dynamic partitioning works, thus we need to make sure our partitioning algorithm can cope with different consistency guarantees.

%Results obtained during the course of this work will be published in top-venue conferences.
%At the moment, a submission to VLDB is being prepared, hence the extensive evaluation work already done.
%We plan to do other submissions to other conferences.
%Interesting publication targets besides VLDB include, for example, OSDI, SOSP, EuroSys, PODC and DISC.

Further results obtained during the course of this work will be published in top-venue conferences.
Aside from our submission to VLDB, we expect more papers coming from our planned contributions.
Interesting publication targets aside from VLDB include, e.g., OSDI, SOSP, EuroSys, Sigmod, PODC and DISC.


%Self TODO: justify small testing periods with all the preparation already done with TPC-H to support testing